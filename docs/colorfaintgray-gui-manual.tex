\documentclass[11pt,a4paper]{article}
\usepackage[utf8]{inputenc}
\usepackage[T1]{fontenc}
\usepackage{geometry}
\usepackage{graphicx}
\usepackage{amsmath}
\usepackage{amsfonts}
\usepackage{amssymb}
\usepackage{xcolor}
\usepackage{listings}
\usepackage{url}
\usepackage{hyperref}
\usepackage{fancyhdr}
\usepackage{booktabs}
\usepackage{float}
\usepackage{subcaption}
\usepackage{enumitem}
\usepackage{parskip}

% Page setup
\geometry{margin=2.5cm}
\hypersetup{colorlinks=true, linkcolor=blue, urlcolor=blue, citecolor=blue}

% Code listings setup
\lstset{
    basicstyle=\ttfamily\small,
    breaklines=true,
    frame=single,
    backgroundcolor=\color{gray!10},
    numbers=left,
    numberstyle=\tiny,
    stepnumber=1,
    showstringspaces=false
}

% Headers and footers
\pagestyle{fancy}
\fancyhf{}
\fancyhead[L]{ColorFaintGray GUI Manual}
\fancyhead[R]{v1.0}
\fancyfoot[C]{\thepage}

\title{\textbf{ColorFaintGray GUI}\\
\large{User Manual and Reference Guide}}
\author{Yogesh Wadadekar\\
\small{Development Year: 2025}}
\date{\today}

\begin{document}

\maketitle

\begin{abstract}
ColorFaintGray GUI is a PyQt6-based graphical interface for GNU Astronomy
Utilities' \texttt{astscript-color-faint-gray} script. This application
enables astronomers to create high-quality color astronomical images from
separate R, G, and B channel FITS files using an intuitive interface with
advanced parameter controls, image caching, comparison tools, and
comprehensive workflow management. The interface addresses the challenge of
visualizing the full dynamic range of astronomical data by implementing the
asinh transformation technique with grayscale backgrounds for faint
features.
\end{abstract}

\tableofcontents
\newpage

\section{Introduction}

\subsection{Overview}

Astronomical images span an enormous dynamic range, from bright stellar cores
with high signal-to-noise ratios to faint diffuse structures barely above
the noise level. Traditional color imaging techniques often saturate bright
regions or lose faint details in black backgrounds. ColorFaintGray GUI
addresses this fundamental challenge by providing an intuitive interface to
the powerful \texttt{astscript-color-faint-gray} algorithm.

The application implements the asinh transformation technique described by
Lupton et al. (2004)\footnote{Lupton, R., Blanton, M. R., Fekete, G., et
al. 2004, PASP, 116, 133} for bright regions while using inverse grayscale
for noisy background areas. This approach, detailed in Infante-Sainz \&
Akhlaghi (2024)\footnote{Infante-Sainz, R. and Akhlaghi, M. 2024, Research
Notes of the AAS, 8, 10}, enables simultaneous visualization of both high
and low surface brightness features within the same image.

\begin{figure}[H]
\centering
% Placeholder for main interface screenshot
\fbox{\parbox{0.8\textwidth}{\centering
\vspace{3cm}
\textit{[Screenshot: Main application interface showing parameter panel,
tabbed workspace, and generated color image]}
\vspace{3cm}
}}
\caption{Main ColorFaintGray GUI interface with parameter controls on the
left and tabbed workspace on the right.}
\label{fig:main-interface}
\end{figure}

\subsection{Key Features}

\begin{itemize}[leftmargin=*]
\item \textbf{Complete Parameter Coverage}: Access to all
astscript-color-faint-gray parameters including input HDUs, weight factors,
minimum values, and zeropoint adjustments
\item \textbf{Intuitive Parameter Control}: Real-time adjustment of asinh
transformation parameters with immediate visual feedback
\item \textbf{Advanced Input Control}: Individual HDU specification, 
weight scaling, minimum clipping, and zeropoint calibration for each channel
\item \textbf{Robust Preset System}: Save, load, and manage parameter
combinations with complete metadata and automatic validation
\item \textbf{Automatic Image Caching}: All generated images cached with
metadata for easy comparison and workflow management
\item \textbf{Multi-Image Comparison}: Side-by-side comparison of up to 6
different parameter combinations
\item \textbf{Command Integration}: View and copy exact command-line
equivalents for reproducible workflows with full parameter sets
\item \textbf{Advanced Visualization}: Support for both traditional color
and innovative grayscale background modes
\item \textbf{Error Handling}: Comprehensive validation and user-friendly
error reporting for all operations
\end{itemize}

\section{Installation and Setup}

\subsection{System Requirements}

\begin{itemize}[leftmargin=*]
\item Python 3.12 or later
\item PyQt6 $\geq$ 6.5.0
\item GNU Astronomy Utilities with \texttt{astscript-color-faint-gray}
\item Scientific Python packages: numpy, astropy, matplotlib, Pillow
\item Sufficient disk space for image caching (recommended: 1GB+)
\end{itemize}

\subsection{Installation Process}

\begin{enumerate}
\item \textbf{Verify GNU Astronomy Utilities}:
\begin{lstlisting}[language=bash]
which astscript-color-faint-gray
astscript-color-faint-gray --help
\end{lstlisting}

\item \textbf{Install Python Dependencies}:
\begin{lstlisting}[language=bash]
pip install -r requirements.txt
\end{lstlisting}

\item \textbf{Launch Application}:
\begin{lstlisting}[language=bash]
python main.py
\end{lstlisting}
\end{enumerate}

\subsection{Configuration}

The application creates a configuration directory at
\texttt{\~{}/.config/astscript-color-faint-gray/} containing:

\begin{itemize}[leftmargin=*]
\item \texttt{config.json}: Application settings and defaults
\item \texttt{presets/}: Saved parameter presets
\item \texttt{command\_history.json}: Command execution history
\end{itemize}

\section{Interface Overview}

\subsection{Main Components}

The ColorFaintGray GUI features a modern tabbed interface with five primary
areas:

\subsubsection{Parameter Panel (Left Side)}

The parameter panel contains all image processing controls organized into
collapsible sections:

\begin{itemize}[leftmargin=*]
\item \textbf{Input Settings}: File selection, HDU specification, weight 
factors, minimum values, and zeropoint adjustments
\item \textbf{Basic Parameters}: Core transformation controls (qbright,
stretch, contrast, gamma)
\item \textbf{Advanced Parameters}: Color/grayscale thresholds, quality
settings, and transformation options
\item \textbf{Output Settings}: File format, destination options, and
quality control
\item \textbf{Command Display}: Real-time command preview with all
parameters
\item \textbf{Action Buttons}: Generate, Reset, comprehensive Preset
management
\end{itemize}

\begin{figure}[H]
\centering
% Placeholder for parameter panel screenshot
\fbox{\parbox{0.4\textwidth}{\centering
\vspace{4cm}
\textit{[Screenshot: Parameter panel showing basic and advanced controls]}
\vspace{4cm}
}}
\caption{Parameter panel with basic and advanced control sections.}
\label{fig:parameter-panel}
\end{figure}

\subsubsection{Tabbed Workspace (Center)}

The main workspace consists of four specialized tabs:

\paragraph{Image Loader Tab}
Handles input file selection and validation:
\begin{itemize}[leftmargin=*]
\item Individual file browsers for R, G, B channels
\item Automatic FITS validation with status indicators
\item File information display with dimensions and statistics
\item Optional thumbnail previews for each channel
\end{itemize}

\paragraph{Preview Tab}
Interactive display of generated images:
\begin{itemize}[leftmargin=*]
\item High-quality rendering with smooth zooming
\item Mouse wheel zoom with cursor-centered scaling
\item Click-and-drag panning for large images
\item Image metadata overlay and statistics
\end{itemize}

\paragraph{Cache Grid Tab}
Thumbnail browser for cached images:
\begin{itemize}[leftmargin=*]
\item Grid layout of automatically generated thumbnails
\item Click thumbnails to restore parameters and view full images
\item Search and filter capabilities
\item Cache management and export tools
\end{itemize}

\paragraph{Compare Tab}
Multi-image comparison interface:
\begin{itemize}[leftmargin=*]
\item Side-by-side display of up to 6 images
\item Synchronized zooming and panning
\item Parameter difference analysis
\item Export comparison data
\end{itemize}

\begin{figure}[H]
\centering
% Placeholder for tabbed interface screenshot
\fbox{\parbox{0.8\textwidth}{\centering
\vspace{3cm}
\textit{[Screenshot: Four-tab interface showing Image Loader, Preview,
Cache Grid, and Compare tabs]}
\vspace{3cm}
}}
\caption{Main tabbed workspace showing all four interface tabs.}
\label{fig:tabs-interface}
\end{figure}

\section{Basic Workflow}

\subsection{Loading Images}

\begin{enumerate}
\item Navigate to the \textbf{Image Loader} tab (active by default)
\item Click \textbf{Browse} buttons to select FITS files for each channel:
   \begin{itemize}
   \item \textbf{R Channel}: Red wavelength data (e.g., r-band, I-band)
   \item \textbf{G Channel}: Green wavelength data (e.g., g-band, V-band)
   \item \textbf{B Channel}: Blue wavelength data (e.g., u-band, B-band)
   \end{itemize}
\item Verify successful loading via green checkmark indicators
\item Review file information in the details panel
\end{enumerate}

\begin{figure}[H]
\centering
% Placeholder for image loader screenshot
\fbox{\parbox{0.7\textwidth}{\centering
\vspace{3cm}
\textit{[Screenshot: Image Loader tab with three loaded FITS files showing
checkmarks and file information]}
\vspace{3cm}
}}
\caption{Image Loader tab with successfully loaded R, G, B channel files.}
\label{fig:image-loader}
\end{figure}

\textbf{Important Considerations}:
\begin{itemize}[leftmargin=*]
\item All images must have identical dimensions
\item Compatible world coordinate systems are recommended
\item Properly calibrated data (flat-fielded, dark-subtracted) yields best
results
\end{itemize}

\subsection{Parameter Adjustment}

The ColorFaintGray GUI provides comprehensive control over all
astscript-color-faint-gray parameters, organized into logical sections for
efficient workflow management.

\subsubsection{Input Settings}

\paragraph{Input Files and HDUs}
Each channel (R, G, B) can specify individual HDU numbers for multi-
extension FITS files:
\begin{itemize}[leftmargin=*]
\item \textbf{HDU Specification}: Individual HDU number for each channel
(default: 0)
\item \textbf{Multi-Extension Support}: Access any image extension within
FITS files
\item \textbf{Flexible Input}: Mix different HDUs from the same or
different files
\end{itemize}

\paragraph{Weight Factors (0.001--1000, default: 1.0)}
Channel-specific scaling factors applied before processing:
\begin{itemize}[leftmargin=*]
\item \textbf{Photometric Calibration}: Compensate for different filter
sensitivities
\item \textbf{Color Balance}: Adjust relative channel contributions
\item \textbf{Exposure Scaling}: Account for different exposure times
\end{itemize}

\paragraph{Minimum Values (auto or manual)}
Pixel value thresholds applied to each channel:
\begin{itemize}[leftmargin=*]
\item \textbf{Noise Clipping}: Remove negative noise artifacts
\item \textbf{Background Subtraction}: Set appropriate zero levels
\item \textbf{Dynamic Range Control}: Optimize for specific brightness
ranges
\end{itemize}

\paragraph{Zeropoint Adjustments (default: 0.0)}
Magnitude zeropoint corrections for photometric accuracy:
\begin{itemize}[leftmargin=*]
\item \textbf{Photometric Calibration}: Apply instrumental magnitude
corrections
\item \textbf{Filter Response}: Compensate for effective wavelength
differences
\item \textbf{Atmospheric Extinction}: Include extinction corrections
\end{itemize}

\subsubsection{Basic Parameters}

\paragraph{qbright (0--100, default: 1.0)}
Controls enhancement of bright features. Higher values make bright regions
(star cores, galaxy centers) more prominent while preserving color balance.

\paragraph{stretch (0--100, default: 1.0)}
Linear stretching parameter for faint features. Increasing this value
reveals more faint background details and diffuse structures.

\paragraph{contrast (0--100, default: 3.0)}
Linear contrast adjustment applied globally. Higher values increase overall
image contrast without affecting the asinh transformation.

\paragraph{gamma (0.1--10.0, default: 0.8)}
Nonlinear brightness adjustment using gamma correction. Values $> 1$
brighten the image, $< 1$ darken it. Applied after linear processing.

\subsubsection{Advanced Parameters}

\paragraph{colorval (auto-estimated)}
Threshold separating color and black regions. Pixels above this value
appear in full color using RGB data.

\paragraph{grayval (auto-estimated)}
Threshold separating black and grayscale regions. Pixels below this value
appear in inverse grayscale.

\paragraph{coloronly (checkbox, default: false)}
When enabled, eliminates grayscale regions entirely. Background appears
either in color or pure black.

\paragraph{forcecolor (checkbox, default: false)}
Forces color mode for all pixels above minimum thresholds, useful for
ensuring consistent color representation across the image.

\paragraph{quality (1--100, default: 95)}
Output image quality setting affecting compression and file size for
formats that support quality adjustment.

\begin{figure}[H]
\centering
% Placeholder for parameter effects illustration
\fbox{\parbox{0.8\textwidth}{\centering
\vspace{4cm}
\textit{[Figure: Grid showing same galaxy with different parameter values
demonstrating effects of qbright, stretch, gamma adjustments]}
\vspace{4cm}
}}
\caption{Parameter effects demonstration: same input with varying qbright,
stretch, and gamma values.}
\label{fig:parameter-effects}
\end{figure}

\subsection{Image Generation}

\begin{enumerate}
\item Ensure all three channel files are loaded and validated
\item Configure input settings if using non-default HDUs, weights, 
minimums, or zeropoints
\item Adjust basic parameters (qbright, stretch, contrast, gamma) as 
desired using the Parameter Panel
\item Set advanced parameters (colorval, grayval, coloronly, etc.) if 
specific output characteristics are required
\item Specify output path and format in the output settings
\item Click the large green \textbf{Generate Image} button
\item Monitor progress in the progress dialog with real-time status updates
\item Review results in the automatically opened Preview tab
\end{enumerate}

\textbf{Important Notes}:
\begin{itemize}[leftmargin=*]
\item Output path specification is now mandatory and automatically validated
\item All parameters are preserved for reproducibility and preset creation
\item Command generation includes complete parameter set for external use
\item Error handling provides detailed feedback for any processing issues
\end{itemize}

Alternative generation methods:
\begin{itemize}[leftmargin=*]
\item Keyboard: F5 or Ctrl+G
\item Menu: Tools $\rightarrow$ Generate Image
\end{itemize}

\subsection{Viewing and Evaluation}

The Preview tab provides comprehensive image viewing capabilities:

\begin{itemize}[leftmargin=*]
\item \textbf{Zoom}: Mouse wheel for cursor-centered scaling
\item \textbf{Pan}: Left-click drag to navigate large images
\item \textbf{Fit}: Right-click menu option to fit image to window
\item \textbf{Information}: Pixel values and coordinates on mouse hover
\end{itemize}

\begin{figure}[H]
\centering
% Placeholder for preview tab screenshot
\fbox{\parbox{0.7\textwidth}{\centering
\vspace{3cm}
\textit{[Screenshot: Preview tab showing generated color image with zoom
controls and information overlay]}
\vspace{3cm}
}}
\caption{Preview tab displaying generated color image with interactive
viewing controls.}
\label{fig:preview-tab}
\end{figure}

\section{Advanced Features}

\subsection{Image Caching System}

All generated images are automatically cached with complete metadata:

\begin{itemize}[leftmargin=*]
\item \textbf{Automatic Storage}: Images saved as high-quality TIFF files
\item \textbf{Thumbnail Generation}: PNG thumbnails for quick browsing
\item \textbf{Parameter Preservation}: Exact parameter values stored
\item \textbf{Command History}: Complete command-line equivalents saved
\item \textbf{Configurable Limits}: Default 25 images (adjustable in
settings)
\end{itemize}

\subsubsection{Cache Grid Navigation}

The Cache Grid tab displays all cached images as interactive thumbnails:

\begin{enumerate}
\item Click any thumbnail to view full-size image in Preview tab
\item Parameters are automatically restored to the Parameter Panel
\item Right-click thumbnails for context menu options:
   \begin{itemize}
   \item Add to Comparison
   \item Copy Command
   \item Show Information
   \item Delete Entry
   \end{itemize}
\end{enumerate}

\begin{figure}[H]
\centering
% Placeholder for cache grid screenshot
\fbox{\parbox{0.8\textwidth}{\centering
\vspace{3cm}
\textit{[Screenshot: Cache Grid tab showing thumbnails of multiple cached
images with timestamps and parameter summaries]}
\vspace{3cm}
}}
\caption{Cache Grid tab displaying thumbnails of automatically cached
images.}
\label{fig:cache-grid}
\end{figure}

\subsection{Multi-Image Comparison}

The Compare tab enables systematic evaluation of different parameter
combinations:

\subsubsection{Adding Images to Comparison}

\begin{itemize}[leftmargin=*]
\item \textbf{From Current}: Tools $\rightarrow$ Add Current to Comparison
(Ctrl+M)
\item \textbf{From Cache}: Right-click thumbnail $\rightarrow$ Add to
Comparison
\item \textbf{Direct Generation}: Generate with different parameters and
add each result
\end{itemize}

\subsubsection{Comparison Features}

\begin{itemize}[leftmargin=*]
\item \textbf{Layout Options}: Grid (2×3), horizontal, or vertical
arrangements
\item \textbf{Synchronized Viewing}: Optional synchronized zoom and pan
across all images
\item \textbf{Parameter Analysis}: Automatic detection and highlighting of
parameter differences
\item \textbf{Export Options}: Save comparison images and analysis data
\end{itemize}

\begin{figure}[H]
\centering
% Placeholder for comparison interface screenshot
\fbox{\parbox{0.8\textwidth}{\centering
\vspace{4cm}
\textit{[Screenshot: Compare tab showing 4 images side-by-side with
parameter difference analysis panel]}
\vspace{4cm}
}}
\caption{Multi-image comparison interface with parameter difference
analysis.}
\label{fig:comparison}
\end{figure}

\subsection{Preset Management}

The ColorFaintGray GUI includes a comprehensive preset management system
that enables efficient reuse of successful parameter combinations with
complete validation and metadata tracking.

\subsubsection{Creating Presets}

\begin{enumerate}
\item Configure all desired parameters in the Parameter Panel (including
input settings, basic parameters, advanced parameters, and output options)
\item Click \textbf{Save Preset} button or use Presets $\rightarrow$ Save
Current as Preset (Ctrl+Shift+S)
\item Enter descriptive name and optional description in the save dialog
\item Preset automatically saved as JSON file with complete parameter set
and metadata
\item Validation ensures all required parameters are captured
\end{enumerate}

\subsubsection{Loading and Managing Presets}

\begin{itemize}[leftmargin=*]
\item \textbf{Quick Load}: Use preset dropdown menu in Parameter Panel for
immediate access
\item \textbf{Load Preset Button}: Browse and load preset files with
automatic validation
\item \textbf{Management Interface}: Access comprehensive preset manager
via Presets $\rightarrow$ Manage Presets (Ctrl+P)
\item \textbf{Automatic Validation}: All loaded presets validated for
parameter completeness and compatibility
\item \textbf{Error Recovery}: Robust handling of corrupted or incomplete
preset files
\end{itemize}

\subsubsection{Preset Features}

\begin{itemize}[leftmargin=*]
\item \textbf{Complete Parameter Coverage}: All parameters saved including
input settings, HDU specifications, weights, and advanced options
\item \textbf{Metadata Preservation}: Creation date, description, and
parameter summaries automatically stored
\item \textbf{Cross-Platform Compatibility}: JSON format ensures presets
work across different systems
\item \textbf{Import/Export}: Share preset collections between users and
installations
\item \textbf{Backup and Recovery}: Automatic backup of preset directory
during operations
\end{itemize}

\subsubsection{Recommended Preset Categories}

\begin{itemize}[leftmargin=*]
\item \textbf{Galaxies - Elliptical}: Optimized for smooth elliptical
galaxy structure with emphasis on outer isophotes
\item \textbf{Galaxies - Spiral}: Balanced parameters for spiral arm
structure and nuclear regions
\item \textbf{Star Fields}: Stellar color preservation with minimal
artifacts in dense fields
\item \textbf{Nebulae - Emission}: Enhanced contrast for H$\alpha$ and
other emission line regions
\item \textbf{Nebulae - Reflection}: Optimized for blue reflection nebulae
and dust lanes
\item \textbf{Deep Field - Survey}: Maximum faint feature enhancement for
survey-depth observations
\item \textbf{Deep Field - Ultra-deep}: Specialized parameters for
ultra-deep field observations
\item \textbf{Custom - High Resolution}: Optimized for high spatial
resolution observations (HST, JWST)
\end{itemize}

\section{Parameter Reference}

\subsection{Mathematical Foundation}

The asinh transformation implements the algorithm described by Lupton et al.
(2004):

\begin{equation}
I_{rgb} = \frac{R + G + B}{3}
\end{equation}

\begin{equation}
f = \frac{\text{asinh}(\alpha \cdot q \cdot I_{rgb})}{\alpha}
\end{equation}

where $\alpha$ is the stretch parameter and $q$ is the qbright parameter.

Each channel is then scaled by:

\begin{equation}
\text{Channel}_{out} = \text{Channel}_{in} \times \frac{f}{I_{rgb}}
\end{equation}

\subsection{Parameter Effects and Ranges}

\begin{table}[H]
\centering
\begin{tabular}{@{}llll@{}}
\toprule
Parameter & Range & Default & Primary Effect \\
\midrule
qbright & 0.001--100 & 1.0 & Bright feature enhancement \\
stretch & 0.001--100 & 1.0 & Faint feature visibility \\
contrast & 0.1--10 & 3.0 & Linear contrast scaling \\
gamma & 0.1--10 & 0.8 & Nonlinear brightness curve \\
R/G/B HDU & 0--999 & 0 & FITS extension selection \\
R/G/B Weight & 0.001--1000 & 1.0 & Channel scaling factor \\
R/G/B Minimum & auto/manual & auto & Noise clipping threshold \\
R/G/B Zeropoint & $\pm$50 & 0.0 & Magnitude calibration \\
colorval & auto/manual & auto & Color/black threshold \\
grayval & auto/manual & auto & Black/gray threshold \\
coloronly & boolean & false & Disable grayscale regions \\
forcecolor & boolean & false & Force color mode \\
quality & 1--100 & 95 & Output image quality \\
\bottomrule
\end{tabular}
\caption{Complete parameter reference with ranges and effects.}
\label{tab:parameters}
\end{table}

\subsection{Object-Specific Recommendations}

\subsubsection{Galaxies and Extended Objects}

Optimal for revealing faint outer structure and tidal streams:

\begin{itemize}[leftmargin=*]
\item qbright: 1.0--2.0 (moderate bright enhancement)
\item stretch: 1.5--3.0 (strong faint feature enhancement)
\item gamma: 0.7--0.9 (slight darkening for better contrast)
\item coloronly: false (preserve grayscale backgrounds)
\end{itemize}

\subsubsection{Star Fields and Clusters}

Emphasizes stellar colors while maintaining photometric accuracy:

\begin{itemize}[leftmargin=*]
\item qbright: 0.5--1.0 (avoid oversaturation)
\item stretch: 0.8--1.5 (moderate faint enhancement)
\item gamma: 0.8--1.0 (preserve stellar brightness relationships)
\item contrast: 2.0--3.0 (maintain color accuracy)
\end{itemize}

\subsubsection{Nebulae and Emission Regions}

Balances emission lines with continuum features:

\begin{itemize}[leftmargin=*]
\item qbright: 1.0--1.5 (reveal internal structure)
\item stretch: 2.0--4.0 (show faint emission)
\item gamma: 0.6--0.8 (darken to enhance contrast)
\item colorval: 0.1--0.5 (adjust color/gray balance)
\end{itemize}

\section{Command-Line Integration}

\subsection{Enhanced Command Display and History}

The Parameter Panel continuously displays the equivalent command-line
invocation with complete parameter coverage:

\begin{lstlisting}[language=bash]
astscript-color-faint-gray -g 0 \
  --rhdu 0 --ghdu 0 --bhdu 0 \
  --rweight 1.0 --gweight 1.0 --bweight 1.0 \
  --rmin auto --gmin auto --bmin auto \
  --rzero 0.0 --gzero 0.0 --bzero 0.0 \
  input_r.fits input_g.fits input_b.fits \
  --qbright 1.5 --stretch 2.0 --contrast 3.0 \
  --gamma 0.8 --quality 95 \
  --output color_image.pdf
\end{lstlisting}

\subsubsection{Command Features}

\begin{itemize}[leftmargin=*]
\item \textbf{Complete Parameter Set}: All GUI parameters reflected in
command line
\item \textbf{Automatic Output Path}: Mandatory output specification
prevents execution errors  
\item \textbf{Real-time Updates}: Command updates immediately as parameters
change
\item \textbf{Copy Functionality}: One-click copying for external use
\item \textbf{Validation}: Commands guaranteed to be syntactically correct
and complete
\end{itemize}

\subsubsection{Command History Access}

\begin{itemize}[leftmargin=*]
\item \textbf{View History}: Tools $\rightarrow$ Command History (Ctrl+H)
\item \textbf{Copy Current}: Tools $\rightarrow$ Copy Current Command
(Ctrl+Shift+C)
\item \textbf{Export History}: Save complete command log for documentation
\item \textbf{Parameter Reconstruction}: Load previous commands to restore
exact parameter states
\end{itemize}

\subsection{Batch Processing Integration}

Commands copied from the GUI can be used for automated processing with
complete parameter specifications:

\begin{lstlisting}[language=bash]
#!/bin/bash
# Batch process multiple image sets using GUI-optimized parameters

# Parameters optimized in GUI for galaxy fields  
BASIC_PARAMS="--qbright 1.5 --stretch 2.2 --gamma 0.8 --contrast 3.0"
INPUT_PARAMS="--rhdu 0 --ghdu 0 --bhdu 0 --rweight 1.0 --gweight 1.0 --bweight 1.0"
ADVANCED_PARAMS="--quality 95"

# Process multiple fields with consistent parameters
for field in field1 field2 field3; do
    echo "Processing ${field}..."
    astscript-color-faint-gray -g 0 \
        $INPUT_PARAMS \
        ${field}_r.fits ${field}_g.fits ${field}_b.fits \
        $BASIC_PARAMS $ADVANCED_PARAMS \
        --output ${field}_color.pdf
done
\end{lstlisting}

\subsubsection{Advanced Batch Processing}

For complex workflows with varying parameters:

\begin{lstlisting}[language=bash]
#!/bin/bash
# Process with different parameters for different object types

# Define parameter sets exported from GUI presets
declare -A PRESETS
PRESETS[galaxy]="--qbright 1.5 --stretch 2.2 --gamma 0.8"
PRESETS[starfield]="--qbright 0.8 --stretch 1.0 --gamma 0.9" 
PRESETS[nebula]="--qbright 1.2 --stretch 2.8 --gamma 0.7"

# Apply appropriate preset based on object type
for object_type in galaxy starfield nebula; do
    for target in ${object_type}_*; do
        [[ -f ${target}_r.fits ]] || continue
        astscript-color-faint-gray -g 0 \
            --rhdu 0 --ghdu 0 --bhdu 0 \
            ${target}_r.fits ${target}_g.fits ${target}_b.fits \
            ${PRESETS[$object_type]} \
            --output ${target}_color.pdf
    done
done
\end{lstlisting}

\section{Color Science and Visualization}

\subsection{Dynamic Range Challenge}

Astronomical images typically exhibit pixel value distributions spanning 4-6
orders of magnitude. The ColorFaintGray approach addresses this by:

\begin{enumerate}
\item \textbf{Bright Regions}: Asinh transformation preserves colors while
preventing saturation
\item \textbf{Intermediate Regions}: Smooth transition to pure black
\item \textbf{Faint Regions}: Inverse grayscale reveals low surface
brightness features
\end{enumerate}

\begin{figure}[H]
\centering
% Placeholder for dynamic range illustration
\fbox{\parbox{0.8\textwidth}{\centering
\vspace{3cm}
\textit{[Figure: Same astronomical field shown with traditional black
background vs. ColorFaintGray technique, highlighting revealed faint
structures]}
\vspace{3cm}
}}
\caption{Dynamic range comparison: traditional color (left) vs.
ColorFaintGray technique (right) showing revealed faint structures.}
\label{fig:dynamic-range}
\end{figure}

\subsection{Color Accuracy Considerations}

\subsubsection{Photometric Calibration}

For scientifically accurate colors:

\begin{itemize}[leftmargin=*]
\item Use properly calibrated input images with consistent zero points
\item Apply appropriate weight values for different filter sensitivities
\item Consider atmospheric extinction corrections
\item Validate colors against known stellar standards
\end{itemize}

\subsubsection{Perceptual Adjustments}

The gamma parameter enables perceptual optimization:

\begin{itemize}[leftmargin=*]
\item Gamma = 1.0: Linear relationship (photometrically accurate)
\item Gamma < 1.0: Emphasizes faint features (better for extended objects)
\item Gamma > 1.0: Enhances bright regions (suitable for stellar fields)
\end{itemize}

\section{Troubleshooting}

\subsection{Common Issues and Solutions}

\subsubsection{Installation Problems}

\textbf{``astscript-color-faint-gray not found''}
\begin{itemize}[leftmargin=*]
\item Verify GNU Astronomy Utilities installation
\item Check PATH environment variable
\item Test manual execution: \texttt{astscript-color-faint-gray --help}
\end{itemize}

\textbf{Python dependency errors}
\begin{itemize}[leftmargin=*]
\item Ensure Python 3.12+ is installed
\item Verify PyQt6 installation: \texttt{python -c "import PyQt6"}
\item Reinstall requirements: \texttt{pip install -r requirements.txt
--upgrade}
\end{itemize}

\subsubsection{File Loading Issues}

\textbf{``Invalid FITS File'' errors}
\begin{itemize}[leftmargin=*]
\item Verify file integrity with \texttt{astropy.io.fits}
\item Check for proper FITS headers and image extensions
\item Ensure files contain image data (not tables or spectra)
\end{itemize}

\textbf{``Dimension Mismatch'' errors}
\begin{itemize}[leftmargin=*]
\item All three input files must have identical pixel dimensions
\item Use \texttt{astinfo} to check image sizes
\item Resample images to common pixel grid if necessary
\end{itemize}

\subsubsection{Generation Failures}

\textbf{Parameter validation errors}
\begin{itemize}[leftmargin=*]
\item Ensure qbright and stretch are not zero
\item Check that all parameters are within valid ranges
\item Reset to defaults and adjust incrementally
\end{itemize}

\textbf{Memory or performance issues}
\begin{itemize}[leftmargin=*]
\item Use smaller test images for parameter optimization
\item Ensure adequate free disk space for temporary files
\item Close other memory-intensive applications
\end{itemize}

\subsection{Performance Optimization}

\subsubsection{Image Size Considerations}

Processing time scales approximately with the square of linear image
dimension:

\begin{table}[H]
\centering
\begin{tabular}{@{}lll@{}}
\toprule
Image Size & Typical Processing Time & Memory Usage \\
\midrule
1000×1000 & 10--30 seconds & 100--200 MB \\
2000×2000 & 1--2 minutes & 400--800 MB \\
4000×4000 & 5--10 minutes & 1.5--3 GB \\
8000×8000 & 20--40 minutes & 6--12 GB \\
\bottomrule
\end{tabular}
\caption{Performance scaling with image size.}
\label{tab:performance}
\end{table}

\subsubsection{Workflow Optimization}

\begin{enumerate}
\item \textbf{Parameter Development}: Use heavily binned or cropped images
for initial parameter exploration
\item \textbf{Cache Utilization}: Leverage automatic caching to avoid
regenerating identical images
\item \textbf{Batch Processing}: Export optimized commands for processing
multiple similar datasets
\end{enumerate}

\section{Best Practices and Workflow}

\subsection{Recommended Workflow}

\begin{enumerate}
\item \textbf{Preparation Phase}:
   \begin{itemize}
   \item Ensure input images are properly calibrated
   \item Verify consistent photometric calibration across channels
   \item Create working copies of original data
   \end{itemize}

\item \textbf{Initial Parameter Exploration}:
   \begin{itemize}
   \item Start with default parameters
   \item Generate initial image to assess overall appearance
   \item Use Cache Grid to track parameter experiments
   \end{itemize}

\item \textbf{Systematic Optimization}:
   \begin{itemize}
   \item Adjust qbright first to optimize bright feature appearance
   \item Fine-tune stretch to reveal desired faint structure level
   \item Use gamma for overall brightness balance
   \item Adjust color/gray thresholds if needed
   \end{itemize}

\item \textbf{Comparison and Validation}:
   \begin{itemize}
   \item Use Compare tab to evaluate different approaches
   \item Generate images with slight parameter variations
   \item Validate colors against known standards
   \end{itemize}

\item \textbf{Final Production}:
   \begin{itemize}
   \item Generate high-quality image with optimized parameters
   \item Save successful parameter set as named preset
   \item Export command for documentation and reproducibility
   \end{itemize}
\end{enumerate}

\subsection{Quality Control Guidelines}

\subsubsection{Scientific Accuracy}

\begin{itemize}[leftmargin=*]
\item Verify that stellar colors match expected values for known star types
\item Check that galaxy colors are consistent with morphological types
\item Ensure that color gradients are astrophysically meaningful
\item Validate against published color-magnitude diagrams when possible
\end{itemize}

\subsubsection{Aesthetic Considerations}

\begin{itemize}[leftmargin=*]
\item Balance between scientific accuracy and visual appeal
\item Avoid over-stretching that introduces noise artifacts
\item Ensure smooth transitions between color, black, and gray regions
\item Consider the intended audience (scientific vs. outreach)
\end{itemize}

\subsection{Documentation and Reproducibility}

\subsubsection{Parameter Recording}

Always document successful parameter combinations:

\begin{itemize}[leftmargin=*]
\item Save parameters as named presets with descriptive names
\item Copy and save the exact command used for each final image
\item Record the rationale for parameter choices
\item Note any special considerations for the dataset
\end{itemize}

\subsubsection{Data Provenance}

Maintain complete records of the image processing chain:

\begin{itemize}[leftmargin=*]
\item Original data sources and observation details
\item Calibration procedures applied before ColorFaintGray processing
\item Complete ColorFaintGray command with all parameters
\item Software versions (Gnuastro, ColorFaintGray GUI, Python, PyQt6)
\end{itemize}

\section{Keyboard Shortcuts}

\begin{table}[H]
\centering
\begin{tabular}{@{}ll@{}}
\toprule
Action & Shortcut \\
\midrule
Generate Image & F5, Ctrl+G \\
Open Images & Ctrl+O \\
Save Image & Ctrl+S \\
Reset Parameters & Ctrl+R \\
Manage Presets & Ctrl+P \\
Save Current as Preset & Ctrl+Shift+S \\
Command History & Ctrl+H \\
Copy Current Command & Ctrl+Shift+C \\
Add to Comparison & Ctrl+M \\
Switch to Image Loader & Ctrl+1 \\
Switch to Preview & Ctrl+2 \\
Switch to Cache Grid & Ctrl+3 \\
Switch to Compare & Ctrl+4 \\
Toggle Parameter Panel & F9 \\
Application Help & F1 \\
\bottomrule
\end{tabular}
\caption{Complete keyboard shortcut reference.}
\label{tab:shortcuts}
\end{table}

\section{Technical Implementation}

\subsection{Software Architecture}

ColorFaintGray GUI is built using a modular architecture:

\begin{itemize}[leftmargin=*]
\item \textbf{Core Module}: Parameter management, command building, image
caching
\item \textbf{GUI Module}: PyQt6-based interface components
\item \textbf{Utils Module}: File handling, error management, configuration
\end{itemize}

\subsection{File Formats and Standards}

\subsubsection{Input Formats}

\begin{itemize}[leftmargin=*]
\item \textbf{FITS}: Primary format, single or multi-extension
\item \textbf{HDU Support}: Flexible HDU specification for complex files
\item \textbf{World Coordinates}: Preserves astrometric information when
present
\end{itemize}

\subsubsection{Output Formats}

\begin{itemize}[leftmargin=*]
\item \textbf{Cache}: High-quality TIFF files for internal storage
\item \textbf{Export}: PDF (default), PNG, JPEG, TIFF options
\item \textbf{Metadata}: JSON format for parameter and command storage
\end{itemize}

\subsection{Performance Characteristics}

\subsubsection{Memory Usage}

\begin{itemize}[leftmargin=*]
\item Base application: 50--100 MB
\item Per image processing: $\sim$3× input file size
\item Cache storage: Configurable, default 25 images
\item Peak memory: Depends on largest processed image
\end{itemize}

\subsubsection{Processing Pipeline}

\begin{enumerate}
\item Input validation and file loading
\item Optional minimum value clipping
\item Weight and zero-point scaling
\item RGB mean calculation (stacking)
\item Asinh transformation application
\item Individual channel scaling
\item Maximum normalization
\item Enhancement (gamma/contrast/bias)
\item Color/gray region separation
\item Final image assembly
\end{enumerate}

\section{Recent Improvements and Version History}

\subsection{Version 1.1 (Current) - Enhanced Parameter Coverage}

The latest version represents a significant expansion of the ColorFaintGray 
GUI capabilities, addressing user feedback and implementing comprehensive 
parameter support:

\subsubsection{Major Enhancements}

\begin{itemize}[leftmargin=*]
\item \textbf{Complete Parameter Coverage}: All astscript-color-faint-gray
parameters now accessible through the GUI interface
\item \textbf{Enhanced Input Control}: Individual HDU specification, weight
factors, minimum value settings, and zeropoint adjustments for each channel
\item \textbf{Robust Preset System}: Completely functional preset save/load
system with comprehensive validation and error handling
\item \textbf{Improved Command Generation}: All commands now include
mandatory output paths with automatic validation
\item \textbf{Advanced Error Handling}: User-friendly error messages and
automatic recovery for common issues
\end{itemize}

\subsubsection{Bug Fixes and Stability Improvements}

\begin{itemize}[leftmargin=*]
\item \textbf{Preset Loading Crash Fix}: Resolved application crash when
loading presets due to missing output file specifications
\item \textbf{Command Validation}: Enhanced command builder ensures all
required parameters are always included
\item \textbf{Parameter Synchronization}: Fixed issues where GUI state and
command generation could become inconsistent
\item \textbf{File Handling}: Improved robustness for various FITS file
formats and structures
\end{itemize}

\subsubsection{User Interface Improvements}

\begin{itemize}[leftmargin=*]
\item \textbf{Organized Parameter Layout}: Logical grouping of input
settings, basic parameters, and advanced options
\item \textbf{Real-time Validation}: Immediate feedback on parameter
validity and conflicts
\item \textbf{Enhanced Tooltips}: Comprehensive help text for all
parameters
\item \textbf{Improved Workflow}: Streamlined parameter adjustment and
preset management processes
\end{itemize}

\subsection{Development Priorities}

Future development focuses on:

\begin{itemize}[leftmargin=*]
\item \textbf{Performance Optimization}: Enhanced processing speed for
large images
\item \textbf{Batch Processing}: GUI support for processing multiple image
sets
\item \textbf{Advanced Visualization}: Additional image analysis and
comparison tools
\item \textbf{Extended Format Support}: Support for additional input and
output formats
\end{itemize}

\section{Acknowledgments and References}

\subsection{Acknowledgments}

ColorFaintGray GUI development is built upon the excellent work of:

\begin{itemize}[leftmargin=*]
\item GNU Astronomy Utilities development team
\item The astscript-color-faint-gray algorithm by Raúl Infante-Sainz and
Mohammad Akhlaghi
\item PyQt6 and the Qt framework for the graphical interface
\item The Python scientific computing ecosystem
\end{itemize}

\subsection{Key References}

\begin{enumerate}
\item Lupton, R., Blanton, M. R., Fekete, G., et al. 2004, ``Preparing Red-
Green-Blue Images from CCD Data,'' PASP, 116, 133
\item Infante-Sainz, R. \& Akhlaghi, M. 2024, ``Gnuastro: Visualizing the
Full Dynamic Range in Color Images,'' Research Notes of the AAS, 8, 10
\item Akhlaghi, M. \& Ichikawa, T. 2015, ``Noise-based Detection and
Segmentation of Nebulous Objects,'' ApJS, 220, 1
\end{enumerate}

\subsection{Software Citation}

When using ColorFaintGray GUI in research, please cite:

\begin{quote}
This work made use of ColorFaintGray GUI v1.0, a Python/PyQt6-based
interface for GNU Astronomy Utilities' astscript-color-faint-gray (Infante-
Sainz \& Akhlaghi 2024). The underlying color image algorithm implements the
asinh transformation technique described by Lupton et al. (2004).
\end{quote}

\section{Appendix}

\subsection{Example Parameter Sets}

\begin{table}[H]
\centering
\small
\begin{tabular}{@{}lcccc@{}}
\toprule
Object Type & qbright & stretch & gamma & Notes \\
\midrule
Elliptical Galaxies & 1.0 & 2.0 & 0.8 & Emphasize outer structure \\
Spiral Galaxies & 1.2 & 1.8 & 0.9 & Balance arms and nucleus \\
Star Forming Regions & 0.8 & 2.5 & 0.7 & Reveal emission structure \\
Globular Clusters & 0.6 & 1.2 & 0.9 & Preserve stellar colors \\
Deep Survey Fields & 1.5 & 3.0 & 0.7 & Maximum faint enhancement \\
Ultra-deep Fields & 2.0 & 3.5 & 0.6 & Extreme faint sensitivity \\
Planetary Nebulae & 1.0 & 2.0 & 0.8 & Balance central star \\
Reflection Nebulae & 0.9 & 2.2 & 0.8 & Preserve blue colors \\
Emission Nebulae & 1.1 & 2.6 & 0.7 & Enhance line emission \\
High-res (HST/JWST) & 0.8 & 1.5 & 0.9 & Preserve fine details \\
\bottomrule
\end{tabular}
\caption{Recommended parameter starting points for different astronomical
objects, with enhanced coverage for modern datasets.}
\label{tab:object-params}
\end{table}

\subsection{Troubleshooting Checklist}

\subsubsection{Before Seeking Help}

\begin{enumerate}
\item Verify all input files are valid FITS format
\item Check that astscript-color-faint-gray executes independently
\item Try with default parameters first
\item Review error messages in dialogs and console output
\item Check available system memory and disk space
\item Ensure Python and PyQt6 versions meet requirements
\end{enumerate}

\subsubsection{Information to Include When Reporting Issues}

\begin{itemize}[leftmargin=*]
\item Operating system and version
\item Python version (\texttt{python --version})
\item PyQt6 version (\texttt{pip show PyQt6})
\item GNU Astronomy Utilities version (\texttt{astscript-color-faint-gray
--version})
\item Input file characteristics (dimensions, format, size)
\item Complete error messages
\item Steps to reproduce the problem
\end{itemize}

\subsection{Configuration File Reference}

The main configuration file (\texttt{config.json}) contains:

\begin{lstlisting}
{
  "app": {
    "last_input_dir": "/path/to/input/directory",
    "last_output_dir": "/path/to/output/directory", 
    "default_output_format": "PDF",
    "cache_size": 25,
    "cache_dir": "cache/",
    "astscript_path": "astscript-color-faint-gray"
  },
  "ui": {
    "parameter_panel_width": 300,
    "grid_thumbnail_size": 150,
    "window_geometry": "saved window position"
  },
  "parameters": {
    "qbright": 1.0,
    "stretch": 1.0,
    "contrast": 3.0,
    "gamma": 0.8
  }
}
\end{lstlisting}

---

\textit{For the most current information and updates, please visit the
project repository and consult the GNU Astronomy Utilities documentation at
\url{https://www.gnu.org/software/gnuastro/}.}

\end{document}
